\section{剑指 Offer II 009. 乘积小于K的子数组}

\href{https://leetcode.cn/problems/ZVAVXX}{Question Link}

描述:给定一个正整数数组 nums和整数 k ,请找出该数组内乘积小于 k 的连续的子数组的个数。

示例:

\begin{lstlisting}
输入: nums = [10,5,2,6], k = 100
输出: 8
解释: 8 个乘积小于 100 的子数组分别为: [10], [5], [2], [6], [10,5], [5,2], [2,6], [5,2,6]。
需要注意的是 [10,5,2] 并不是乘积小于100的子数组。
\end{lstlisting}

%------------------------------------------------

\subsection{解析}

滑动窗口典型题目,从前到后处理所有的 $nums[hi]$,使用变量 product 记录当前窗口的乘积,使用 lo 和 hi 分别记录当前窗口的左右端点([lo, hi])。

当 product >= k 时,将左端点 lo 右移,同时使 product/=nums[lo],消除左端点的贡献,这样,对于右端点 hi,我们就可以得到其对应的左端点 lo,从而获得以右端点 hi 为结尾的合法的子数组个数为 hi-lo+1。

\subsection*{Code:}

\begin{lstlisting}[language=C++]
class Solution {
 public:
  int numSubarrayProductLessThanK(vector<int>& nums, int k) {
    int lo = 0, hi = 0;
    int product = 1, count = 0;
    while (hi < nums.size()) {
      product *= nums[hi];
      while (lo <= hi && product >= k) {
        product /= nums[lo++];
      }
      count += hi - lo + 1;
      hi++;
    }
    return count;
  }
};
\end{lstlisting}
